%% Generated by Sphinx.
\def\sphinxdocclass{report}
\documentclass[letterpaper,10pt,english]{sphinxmanual}
\ifdefined\pdfpxdimen
   \let\sphinxpxdimen\pdfpxdimen\else\newdimen\sphinxpxdimen
\fi \sphinxpxdimen=.75bp\relax

\PassOptionsToPackage{warn}{textcomp}
\usepackage[utf8]{inputenc}
\ifdefined\DeclareUnicodeCharacter
% support both utf8 and utf8x syntaxes
  \ifdefined\DeclareUnicodeCharacterAsOptional
    \def\sphinxDUC#1{\DeclareUnicodeCharacter{"#1}}
  \else
    \let\sphinxDUC\DeclareUnicodeCharacter
  \fi
  \sphinxDUC{00A0}{\nobreakspace}
  \sphinxDUC{2500}{\sphinxunichar{2500}}
  \sphinxDUC{2502}{\sphinxunichar{2502}}
  \sphinxDUC{2514}{\sphinxunichar{2514}}
  \sphinxDUC{251C}{\sphinxunichar{251C}}
  \sphinxDUC{2572}{\textbackslash}
\fi
\usepackage{cmap}
\usepackage[T1]{fontenc}
\usepackage{amsmath,amssymb,amstext}
\usepackage{babel}



\usepackage{times}
\expandafter\ifx\csname T@LGR\endcsname\relax
\else
% LGR was declared as font encoding
  \substitutefont{LGR}{\rmdefault}{cmr}
  \substitutefont{LGR}{\sfdefault}{cmss}
  \substitutefont{LGR}{\ttdefault}{cmtt}
\fi
\expandafter\ifx\csname T@X2\endcsname\relax
  \expandafter\ifx\csname T@T2A\endcsname\relax
  \else
  % T2A was declared as font encoding
    \substitutefont{T2A}{\rmdefault}{cmr}
    \substitutefont{T2A}{\sfdefault}{cmss}
    \substitutefont{T2A}{\ttdefault}{cmtt}
  \fi
\else
% X2 was declared as font encoding
  \substitutefont{X2}{\rmdefault}{cmr}
  \substitutefont{X2}{\sfdefault}{cmss}
  \substitutefont{X2}{\ttdefault}{cmtt}
\fi


\usepackage[Bjarne]{fncychap}
\usepackage[,numfigreset=1,mathnumfig]{sphinx}

\fvset{fontsize=\small}
\usepackage{geometry}


% Include hyperref last.
\usepackage{hyperref}
% Fix anchor placement for figures with captions.
\usepackage{hypcap}% it must be loaded after hyperref.
% Set up styles of URL: it should be placed after hyperref.
\urlstyle{same}


\usepackage{sphinxmessages}




\title{V-Berth Volume}
\date{May 27, 2021}
\release{}
\author{The Jupyter Book community}
\newcommand{\sphinxlogo}{\vbox{}}
\renewcommand{\releasename}{}
\makeindex
\begin{document}

\pagestyle{empty}
\sphinxmaketitle
\pagestyle{plain}
\sphinxtableofcontents
\pagestyle{normal}
\phantomsection\label{\detokenize{index::doc}}


\sphinxAtStartPar
The objective is to compute the volume of a water tank under the V\sphinxhyphen{}berth on a boat.

\sphinxAtStartPar
The problem we have is that it’s really difficult to get accurate measurements.
We needs a spreadsheet\sphinxhyphen{}like capability to
\begin{itemize}
\item {} 
\sphinxAtStartPar
Define a computation. In this case, a rather complex one.

\item {} 
\sphinxAtStartPar
Provide inputs.

\item {} 
\sphinxAtStartPar
See results. We want immediate gratification here. A “compute all cells” kind of thing
that will rework the math with our new measurements.

\end{itemize}

\sphinxAtStartPar
Additionally, we want to be able to diagnose errors in the computation, which
means a bunch of cells with hidden formulae aren’t optimal. We want to see the
details, including some demonstration that they work. You know, test cases.

\sphinxAtStartPar
For the non\sphinxhyphen{}nautical, the V\sphinxhyphen{}berth is the v\sphinxhyphen{}shaped bedroom built into the bow of
the boat. This is a complex shape, flat on the top, but V\sphinxhyphen{}shaped on the bottom.
It sweeps up at the forward end, almost coming to a point.

\sphinxAtStartPar
This isn’t easy to visualize. The following sketch views the tank from the aft end
looking forward, toward the pointy end of the boat.
It shows how there’s a triangular face aft, and a much smaller triangular face forward.
The top tapers from the aft end, and is a kind of truncated triangle.
The sides slope down along the inside of the boat’s hull.

\sphinxAtStartPar
\sphinxincludegraphics{{IMG_0079}.png}

\sphinxAtStartPar
This diagram is facing forward, showing the large triangle at the aft end of the tank.
The sizes were preliminary measurements, later refined as the top was cut away to gain
access to the interior.

\sphinxAtStartPar
We can describe the space in a number of ways, leading us to three models for the volume:
\begin{itemize}
\item {} 
\sphinxAtStartPar
\sphinxstylestrong{Regular Triangular Prism}. While the space tapers from the large aft end to almost a point at the forward end,
we can use the midpoints along this axis to define
a triangular prism that should be equivalent to the irregular shape.

\item {} 
\sphinxAtStartPar
\sphinxstylestrong{Regular Tetrahedron}. While the tank isn’t really regular, we can take the mean lengths of the six edges,
and use this to describe a regular tetrahedron that should have a similar volume.

\item {} 
\sphinxAtStartPar
\sphinxstylestrong{Irregular Triangular Prism}. The tank is a prism that tapers from aft to forward. We can describe this taper
as a function of the distance along the fore\sphinxhyphen{}and\sphinxhyphen{}aft axis of the tank.

\end{itemize}

\sphinxAtStartPar
The differences, it turns out, are relatively minor. The simpler math of a triangular prism
is accurate enough for our purposes. The other two models serve as confirmation of the volume.

\sphinxAtStartPar
While a single estimate of the volume is necessary, it’s not sufficient.
As the top of the structure was removed, it became easier to get more accurate measuremewnts.
What would help is a closed form function that transforms a given collection measurements
into a volume estimate.

\sphinxAtStartPar
It helps to have the computation in the form of a Jupyter Notebook.
We can change the measurements and see resulting change in the volume.
This lets us answer a number of design tradeoff questions regarding ways
to assemble new water tanks or flexible bladders to fill the leaky old aluminum tank that’s in there.

\sphinxAtStartPar
We’ll start with the Triangular Prism, because it involves the least math.


\chapter{Triangular Prism}
\label{\detokenize{prism:triangular-prism}}\label{\detokenize{prism::doc}}
\sphinxAtStartPar
A prism is a solid with two triangular faces joined by three rectangular faces. Referring back to the diagram, this is clearly not a good description of the space. The aft and and forward end are triangular, but the remaining three faces are not rectangles; the remaining faces are irregular quadrilaterals.

\sphinxAtStartPar
The volume of a triangular prism is the area of the triangle, \(\frac{1}{2} h \times w\), times the overall length of the prism, \(l\).
\begin{equation*}
\begin{split}
V = \frac{\frac{h \times w}{2} \times l}{231 \textbf{ cu in/gal}}
\end{split}
\end{equation*}
\sphinxAtStartPar
Our water tank has a large triangle aft and a small triangle forward. If we take the mean between these two extreme sizes, this should describe the space.

\begin{sphinxVerbatim}[commandchars=\\\{\}]
\PYG{k+kn}{from} \PYG{n+nn}{myst\PYGZus{}nb} \PYG{k+kn}{import} \PYG{n}{glue}
\end{sphinxVerbatim}

\begin{sphinxVerbatim}[commandchars=\\\{\}]
\PYG{k+kn}{from} \PYG{n+nn}{sympy} \PYG{k+kn}{import} \PYG{o}{*}
\PYG{n}{h}\PYG{p}{,} \PYG{n}{w}\PYG{p}{,} \PYG{n}{l}\PYG{p}{,} \PYG{n}{V} \PYG{o}{=} \PYG{n}{symbols}\PYG{p}{(}\PYG{l+s+s1}{\PYGZsq{}}\PYG{l+s+s1}{h w l V}\PYG{l+s+s1}{\PYGZsq{}}\PYG{p}{)}
\PYG{n}{glue}\PYG{p}{(}\PYG{l+s+s2}{\PYGZdq{}}\PYG{l+s+s2}{Vol1}\PYG{l+s+s2}{\PYGZdq{}}\PYG{p}{,} \PYG{n}{Eq}\PYG{p}{(}\PYG{n}{V}\PYG{p}{,} \PYG{p}{(}\PYG{n}{h} \PYG{o}{*} \PYG{n}{w}\PYG{p}{)} \PYG{o}{/} \PYG{l+m+mi}{2} \PYG{o}{*} \PYG{n}{l} \PYG{o}{/} \PYG{l+m+mi}{231}\PYG{p}{,} \PYG{n}{evaluate}\PYG{o}{=}\PYG{k+kc}{False}\PYG{p}{)}\PYG{p}{)}
\end{sphinxVerbatim}
\begin{equation*}
\begin{split}\displaystyle V = \frac{h l w}{462}\end{split}
\end{equation*}
\sphinxAtStartPar
Here’s a slightly simplified volume of a triangular prism, \(V\), including the conversion from cubic inches to gallons.
\begin{equation}\label{equation:index:prism}
\begin{split}\displaystyle V = \frac{h l w}{462}\end{split}
\end{equation}
\sphinxAtStartPar
Given a width, \(w\), height, \(h\), and the overall length, \(l\), we know the volume of the space, \(V\).

\sphinxAtStartPar
Since the triangles are isoscolese, we’ve measured their base edge and height. We often call the base edge the “width” of the triangle in the computations that follow.
If necessary, we can compute the other two edges, \(e = \sqrt{h^2+{\frac{w}{2}}^2}\).


\section{Middling the triangles}
\label{\detokenize{prism:middling-the-triangles}}
\sphinxAtStartPar
For this model of the space, we’ll take the middle of the widths and the middle of the heights to describe a regular triangular prism.

\sphinxAtStartPar
The height at the  midpoint, \(h_m\), and the width at the midpoint, \(w_m\), are as follows:

\begin{sphinxVerbatim}[commandchars=\\\{\}]
\PYG{n}{var}\PYG{p}{(}\PYG{l+s+s2}{\PYGZdq{}}\PYG{l+s+s2}{h\PYGZus{}f, h\PYGZus{}a, w\PYGZus{}f, w\PYGZus{}a, l\PYGZus{}fa}\PYG{l+s+s2}{\PYGZdq{}}\PYG{p}{)}
\end{sphinxVerbatim}

\begin{sphinxVerbatim}[commandchars=\\\{\}]
(h\PYGZus{}f, h\PYGZus{}a, w\PYGZus{}f, w\PYGZus{}a, l\PYGZus{}fa)
\end{sphinxVerbatim}

\begin{sphinxVerbatim}[commandchars=\\\{\}]
\PYG{n}{h\PYGZus{}m} \PYG{o}{=} \PYG{n}{Rational}\PYG{p}{(}\PYG{l+m+mi}{1}\PYG{p}{,} \PYG{l+m+mi}{2}\PYG{p}{)}\PYG{o}{*}\PYG{p}{(}\PYG{n}{h\PYGZus{}f} \PYG{o}{+} \PYG{n}{h\PYGZus{}a}\PYG{p}{)}
\PYG{n}{w\PYGZus{}m} \PYG{o}{=} \PYG{n}{Rational}\PYG{p}{(}\PYG{l+m+mi}{1}\PYG{p}{,} \PYG{l+m+mi}{2}\PYG{p}{)}\PYG{o}{*}\PYG{p}{(}\PYG{n}{w\PYGZus{}f} \PYG{o}{+} \PYG{n}{w\PYGZus{}a}\PYG{p}{)}
\end{sphinxVerbatim}

\sphinxAtStartPar
It can help to see these expressions type set and simplified. Here’s what the midpoints look like.

\begin{sphinxVerbatim}[commandchars=\\\{\}]
\PYG{n}{h\PYGZus{}m}
\end{sphinxVerbatim}
\begin{equation*}
\begin{split}\displaystyle \frac{h_{a}}{2} + \frac{h_{f}}{2}\end{split}
\end{equation*}
\begin{sphinxVerbatim}[commandchars=\\\{\}]
\PYG{n}{w\PYGZus{}m}
\end{sphinxVerbatim}
\begin{equation*}
\begin{split}\displaystyle \frac{w_{a}}{2} + \frac{w_{f}}{2}\end{split}
\end{equation*}

\section{Volume}
\label{\detokenize{prism:volume}}
\sphinxAtStartPar
The mid\sphinxhyphen{}point volume, \(V_m\), is computed from the midpoint height and width, \(h_m\) and \(w_m\), and the overall length, \(l_{fa}\). We can expand the midpoint values with the fore\sphinxhyphen{}and\sphinxhyphen{}aft sizes
to get an overall computation given the five measurements we have.

\begin{sphinxVerbatim}[commandchars=\\\{\}]
\PYG{n}{V\PYGZus{}m} \PYG{o}{=} \PYG{p}{(}\PYG{p}{(}\PYG{n}{h\PYGZus{}m} \PYG{o}{*} \PYG{n}{w\PYGZus{}m}\PYG{p}{)}\PYG{o}{/}\PYG{l+m+mi}{2} \PYG{o}{*} \PYG{n}{l\PYGZus{}fa}\PYG{p}{)} \PYG{o}{/} \PYG{l+m+mi}{231}
\PYG{n}{ratsimp}\PYG{p}{(}\PYG{n}{V\PYGZus{}m}\PYG{p}{)}
\end{sphinxVerbatim}
\begin{equation*}
\begin{split}\displaystyle \frac{h_{a} l_{fa} w_{a}}{1848} + \frac{h_{a} l_{fa} w_{f}}{1848} + \frac{h_{f} l_{fa} w_{a}}{1848} + \frac{h_{f} l_{fa} w_{f}}{1848}\end{split}
\end{equation*}
\sphinxAtStartPar
This form, while kind of confusing\sphinxhyphen{}looking, has various \(h \times w \times l\) terms, giving us confidence that this will be a proper volume computation in cubic inches based on all three dimensions.


\section{Measurements}
\label{\detokenize{prism:measurements}}
\sphinxAtStartPar
The actual dimensions are as follows.
(And, yes, these differ slightly from sketches shown earlier, these are actual sizes.) These are subject to change, so we’ve collected them in one plce.

\begin{sphinxVerbatim}[commandchars=\\\{\}]
\PYG{n}{measured} \PYG{o}{=} \PYG{p}{\PYGZob{}}
    \PYG{c+c1}{\PYGZsh{} Forward triangle, in inches}
    \PYG{l+s+s2}{\PYGZdq{}}\PYG{l+s+s2}{h\PYGZus{}f}\PYG{l+s+s2}{\PYGZdq{}}\PYG{p}{:} \PYG{l+m+mi}{8}\PYG{p}{,}
    \PYG{l+s+s2}{\PYGZdq{}}\PYG{l+s+s2}{w\PYGZus{}f}\PYG{l+s+s2}{\PYGZdq{}}\PYG{p}{:} \PYG{l+m+mi}{10} \PYG{o}{+} \PYG{n}{Rational}\PYG{p}{(}\PYG{l+m+mi}{1}\PYG{p}{,} \PYG{l+m+mi}{2}\PYG{p}{)}\PYG{p}{,}

    \PYG{c+c1}{\PYGZsh{} Aft triangle, in inches}
    \PYG{l+s+s2}{\PYGZdq{}}\PYG{l+s+s2}{h\PYGZus{}a}\PYG{l+s+s2}{\PYGZdq{}}\PYG{p}{:} \PYG{l+m+mi}{27}\PYG{p}{,}
    \PYG{l+s+s2}{\PYGZdq{}}\PYG{l+s+s2}{w\PYGZus{}a}\PYG{l+s+s2}{\PYGZdq{}}\PYG{p}{:} \PYG{l+m+mi}{48}\PYG{p}{,}

    \PYG{c+c1}{\PYGZsh{} Overall length from forward to aft, in inches.}
    \PYG{l+s+s2}{\PYGZdq{}}\PYG{l+s+s2}{l\PYGZus{}fa}\PYG{l+s+s2}{\PYGZdq{}}\PYG{p}{:} \PYG{l+m+mi}{46}\PYG{p}{,}
\PYG{p}{\PYGZcb{}}
\end{sphinxVerbatim}

\sphinxAtStartPar
We can substitute the measured values into the volume formula to compute the volume of the space. We’ll need to reformulate this slightly, since it’s a complex\sphinxhyphen{}looking fraction.

\begin{sphinxVerbatim}[commandchars=\\\{\}]
\PYG{n}{V\PYGZus{}m}\PYG{o}{.}\PYG{n}{subs}\PYG{p}{(}\PYG{n}{measured}\PYG{p}{)}
\end{sphinxVerbatim}
\begin{equation*}
\begin{split}\displaystyle \frac{4485}{88}\end{split}
\end{equation*}
\begin{sphinxVerbatim}[commandchars=\\\{\}]
\PYG{l+s+sa}{f}\PYG{l+s+s2}{\PYGZdq{}}\PYG{l+s+si}{\PYGZob{}}\PYG{n}{floor}\PYG{p}{(}\PYG{n}{V\PYGZus{}m}\PYG{o}{.}\PYG{n}{subs}\PYG{p}{(}\PYG{n}{measured}\PYG{p}{)}\PYG{p}{)}\PYG{l+s+si}{\PYGZcb{}}\PYG{l+s+s2}{ }\PYG{l+s+si}{\PYGZob{}}\PYG{n}{frac}\PYG{p}{(}\PYG{n}{V\PYGZus{}m}\PYG{o}{.}\PYG{n}{subs}\PYG{p}{(}\PYG{n}{measured}\PYG{p}{)}\PYG{p}{)}\PYG{l+s+si}{\PYGZcb{}}\PYG{l+s+s2}{ gallons}\PYG{l+s+s2}{\PYGZdq{}}
\end{sphinxVerbatim}

\begin{sphinxVerbatim}[commandchars=\\\{\}]
\PYGZsq{}50 85/88 gallons\PYGZsq{}
\end{sphinxVerbatim}

\sphinxAtStartPar
We can define a formal equation for volume given the five measurements. This is something we can use to recompute volume as the measurements evolve.

\begin{sphinxVerbatim}[commandchars=\\\{\}]
\PYG{n}{glue}\PYG{p}{(}\PYG{l+s+s2}{\PYGZdq{}}\PYG{l+s+s2}{Vol\PYGZus{}m}\PYG{l+s+s2}{\PYGZdq{}}\PYG{p}{,} \PYG{n}{V\PYGZus{}m}\PYG{o}{.}\PYG{n}{evalf}\PYG{p}{(}\PYG{l+m+mi}{3}\PYG{p}{)}\PYG{p}{)}
\end{sphinxVerbatim}
\begin{equation*}
\begin{split}\displaystyle 0.00216 l_{fa} \left(0.5 h_{a} + 0.5 h_{f}\right) \left(0.5 w_{a} + 0.5 w_{f}\right)\end{split}
\end{equation*}
\sphinxAtStartPar
Here’s a the volume of a triangular prism, including the conversion from cubic inches to gallons.
\begin{equation}\label{equation:index:prism_mid}
\begin{split}\displaystyle 0.00216 l_{fa} \left(0.5 h_{a} + 0.5 h_{f}\right) \left(0.5 w_{a} + 0.5 w_{f}\right)\end{split}
\end{equation}
\sphinxAtStartPar
Given the the following:
\begin{itemize}
\item {} 
\sphinxAtStartPar
width, \(w_f\), and height, \(h_f\), of the forward triangle,

\item {} 
\sphinxAtStartPar
width, \(w_a\), and height, \(h_a\), of the aft triangle,

\item {} 
\sphinxAtStartPar
and the and the overall length, \(l\).

\end{itemize}

\sphinxAtStartPar
Finally, here’s the number of gallons as a decimal number.

\begin{sphinxVerbatim}[commandchars=\\\{\}]
\PYG{n}{V\PYGZus{}m}\PYG{o}{.}\PYG{n}{evalf}\PYG{p}{(}\PYG{l+m+mi}{4}\PYG{p}{,} \PYG{n}{measured}\PYG{p}{)}
\end{sphinxVerbatim}
\begin{equation*}
\begin{split}\displaystyle 50.97\end{split}
\end{equation*}
\sphinxAtStartPar
We’ll use this as a baseline to compare to the other estimates.  Next, we’ll treat the volume as a regular tetrahedron, also. The math is a little more complex: we’ll need to deduce the lengths of some edges from the available height and width values.


\chapter{Regular Tetrahedron}
\label{\detokenize{tetrahedron:regular-tetrahedron}}\label{\detokenize{tetrahedron::doc}}
\sphinxAtStartPar
A tetrahedron is a structure with four triangular faces.

\sphinxAtStartPar
Because the tank is flattened at the forward end, we could define a full\sphinxhyphen{}sized tetrahedron, and
then truncate a tiny tetrahedron from the front to better estimate the remaining volume.

\sphinxAtStartPar
A symmetric (or regular) Tetrahedron has a volume, \(V\), based on the size of the edges, \(a\).
\begin{equation*}
\begin{split}
V = \frac{a^3}{6\sqrt{2}}
\end{split}
\end{equation*}
\sphinxAtStartPar
In our case, the actual tank in question has a number of different edge lengths:
\begin{itemize}
\item {} 
\sphinxAtStartPar
There are three edges on the aft triangular face. We know the width and height, and can compute the other two edges.

\item {} 
\sphinxAtStartPar
Of the three edges on the top triangular face, it shares an edge with the aft face. We know the width and height here, too, and can compute the remaining edges.

\item {} 
\sphinxAtStartPar
Finally, the seam along the keel is the sixth and final edge. We can compute this knowing the the top and aft form a right triangle. The hypotenuse of this triangle is the bottom seam.

\end{itemize}

\sphinxAtStartPar
We have accurate measurements of one edge and two heights. From these we can compute the others.

\begin{sphinxVerbatim}[commandchars=\\\{\}]
\PYG{k+kn}{from} \PYG{n+nn}{myst\PYGZus{}nb} \PYG{k+kn}{import} \PYG{n}{glue}
\end{sphinxVerbatim}

\sphinxAtStartPar
The measurements include some heights of isoscoles triangles, from which we need to work out the lengths of the edges.

\sphinxAtStartPar
The aft face, for example is 48 inches across the base, \(w\), and 27 tall, \(h\). We can divide this into two right triangles, and compute the remaining edge, \(e\), length.
\begin{equation*}
\begin{split}
e = \sqrt{{\tfrac{w}{2}}^2 + h^2}
\end{split}
\end{equation*}

\section{Computing the six edges}
\label{\detokenize{tetrahedron:computing-the-six-edges}}
\sphinxAtStartPar
We’ve measured one edge and two heights:
\begin{itemize}
\item {} 
\sphinxAtStartPar
The the top of the aft section edge is 48 inches.

\item {} 
\sphinxAtStartPar
The height of the base is 27 inches.

\item {} 
\sphinxAtStartPar
The height of the top is 46 inces.

\end{itemize}

\sphinxAtStartPar
From these, we can compute the remaining edges applying the rule \(e^2 = {\frac{w}{2}}^2 + h^2\). This works because each isoscolese triangle is two right triangles.

\begin{sphinxVerbatim}[commandchars=\\\{\}]
\PYG{k+kn}{from} \PYG{n+nn}{sympy} \PYG{k+kn}{import} \PYG{o}{*}
\PYG{n}{a\PYGZus{}1}\PYG{p}{,} \PYG{n}{a\PYGZus{}2}\PYG{p}{,} \PYG{n}{a\PYGZus{}3}\PYG{p}{,} \PYG{n}{a\PYGZus{}4}\PYG{p}{,} \PYG{n}{a\PYGZus{}5}\PYG{p}{,} \PYG{n}{a\PYGZus{}6} \PYG{o}{=} \PYG{n}{symbols}\PYG{p}{(}\PYG{l+s+s1}{\PYGZsq{}}\PYG{l+s+s1}{a\PYGZus{}1 a\PYGZus{}2 a\PYGZus{}3 a\PYGZus{}4 a\PYGZus{}5 a\PYGZus{}6}\PYG{l+s+s1}{\PYGZsq{}}\PYG{p}{)}
\end{sphinxVerbatim}

\begin{sphinxVerbatim}[commandchars=\\\{\}]
\PYG{c+c1}{\PYGZsh{} The \PYGZdq{}width\PYGZdq{} (actually the top) of the aft triangle.}
\PYG{n}{a\PYGZus{}1} \PYG{o}{=} \PYG{l+m+mi}{48}

\PYG{c+c1}{\PYGZsh{} The two sides of the aft triangle, given a height of 27\PYGZdq{}.}
\PYG{n}{a\PYGZus{}2} \PYG{o}{=} \PYG{n}{sqrt}\PYG{p}{(}\PYG{p}{(}\PYG{n}{Rational}\PYG{p}{(}\PYG{l+m+mi}{1}\PYG{p}{,} \PYG{l+m+mi}{2}\PYG{p}{)}\PYG{o}{*}\PYG{n}{a\PYGZus{}1}\PYG{p}{)}\PYG{o}{*}\PYG{o}{*}\PYG{l+m+mi}{2} \PYG{o}{+} \PYG{n}{S}\PYG{p}{(}\PYG{l+m+mi}{27}\PYG{p}{)}\PYG{o}{*}\PYG{o}{*}\PYG{l+m+mi}{2}\PYG{p}{)}
\PYG{n}{a\PYGZus{}3} \PYG{o}{=} \PYG{n}{a\PYGZus{}2}
\end{sphinxVerbatim}

\sphinxAtStartPar
If the top were not truncated, it would be a triangle with a base of 48 inches, and height that’s a little more than 46 inches. We’ll stick with the truncated measurement of 46 inches, rather than estimate the full, untruncated height.

\begin{sphinxVerbatim}[commandchars=\\\{\}]
\PYG{c+c1}{\PYGZsh{} The two edges of the top, given a height of 46\PYGZdq{}.}
\PYG{n}{a\PYGZus{}4} \PYG{o}{=} \PYG{n}{sqrt}\PYG{p}{(}\PYG{p}{(}\PYG{n}{Rational}\PYG{p}{(}\PYG{l+m+mi}{1}\PYG{p}{,} \PYG{l+m+mi}{2}\PYG{p}{)}\PYG{o}{*}\PYG{n}{a\PYGZus{}1}\PYG{p}{)}\PYG{o}{*}\PYG{o}{*}\PYG{l+m+mi}{2} \PYG{o}{+} \PYG{n}{S}\PYG{p}{(}\PYG{l+m+mi}{46}\PYG{p}{)}\PYG{o}{*}\PYG{o}{*}\PYG{l+m+mi}{2}\PYG{p}{)}
\PYG{n}{a\PYGZus{}5} \PYG{o}{=} \PYG{n}{a\PYGZus{}4}
\end{sphinxVerbatim}

\sphinxAtStartPar
The seam along the bottom is the hypotenuse of a 46” by 27” right triangle.

\begin{sphinxVerbatim}[commandchars=\\\{\}]
\PYG{n}{a\PYGZus{}6} \PYG{o}{=} \PYG{n}{sqrt}\PYG{p}{(}\PYG{n}{S}\PYG{p}{(}\PYG{l+m+mi}{27}\PYG{p}{)}\PYG{o}{*}\PYG{o}{*}\PYG{l+m+mi}{2} \PYG{o}{+} \PYG{n}{S}\PYG{p}{(}\PYG{l+m+mi}{46}\PYG{p}{)}\PYG{o}{*}\PYG{o}{*}\PYG{l+m+mi}{2}\PYG{p}{)}
\end{sphinxVerbatim}

\sphinxAtStartPar
We can take the mean of these edges to estimate a regular tetrahedron that has a similar volume.

\begin{sphinxVerbatim}[commandchars=\\\{\}]
\PYG{n}{m} \PYG{o}{=} \PYG{p}{(}\PYG{n}{a\PYGZus{}1} \PYG{o}{+} \PYG{n}{a\PYGZus{}2} \PYG{o}{+} \PYG{n}{a\PYGZus{}3} \PYG{o}{+} \PYG{n}{a\PYGZus{}4} \PYG{o}{+} \PYG{n}{a\PYGZus{}5} \PYG{o}{+} \PYG{n}{a\PYGZus{}6}\PYG{p}{)}\PYG{o}{/}\PYG{l+m+mi}{6}
\end{sphinxVerbatim}

\begin{sphinxVerbatim}[commandchars=\\\{\}]
\PYG{n}{glue}\PYG{p}{(}\PYG{l+s+s2}{\PYGZdq{}}\PYG{l+s+s2}{edge}\PYG{l+s+s2}{\PYGZdq{}}\PYG{p}{,} \PYG{n}{radsimp}\PYG{p}{(}\PYG{n}{m}\PYG{p}{)}\PYG{p}{)}
\PYG{n}{glue}\PYG{p}{(}\PYG{l+s+s2}{\PYGZdq{}}\PYG{l+s+s2}{edge\PYGZus{}f}\PYG{l+s+s2}{\PYGZdq{}}\PYG{p}{,} \PYG{n}{m}\PYG{o}{.}\PYG{n}{evalf}\PYG{p}{(}\PYG{l+m+mi}{4}\PYG{p}{)}\PYG{p}{)}
\end{sphinxVerbatim}
\begin{equation*}
\begin{split}\displaystyle 8 + \frac{\sqrt{2845} + 6 \sqrt{145} + 4 \sqrt{673}}{6}\end{split}
\end{equation*}\begin{equation*}
\begin{split}\displaystyle 46.23\end{split}
\end{equation*}
\sphinxAtStartPar
This suggests an edge length of \DUrole{pasted-inline}{\(\displaystyle 8 + \frac{\sqrt{2845} + 6 \sqrt{145} + 4 \sqrt{673}}{6}\)},
which is approximately \DUrole{pasted-text}{46.23} inches.

\sphinxAtStartPar
The volume needs to be converted from cubic inches to gallons, using the constant \(231 \frac{\textbf{cu in}}{\textbf{gal}}\)

\begin{sphinxVerbatim}[commandchars=\\\{\}]
\PYG{n}{V\PYGZus{}r} \PYG{o}{=} \PYG{n}{m}\PYG{o}{*}\PYG{o}{*}\PYG{l+m+mi}{3} \PYG{o}{/} \PYG{p}{(}\PYG{l+m+mi}{6} \PYG{o}{*} \PYG{n}{sqrt}\PYG{p}{(}\PYG{l+m+mi}{2}\PYG{p}{)}\PYG{p}{)} \PYG{o}{/} \PYG{l+m+mi}{231}
\end{sphinxVerbatim}

\begin{sphinxVerbatim}[commandchars=\\\{\}]
\PYG{n}{var}\PYG{p}{(}\PYG{l+s+s2}{\PYGZdq{}}\PYG{l+s+s2}{V}\PYG{l+s+s2}{\PYGZdq{}}\PYG{p}{)}
\PYG{n}{glue}\PYG{p}{(}\PYG{l+s+s2}{\PYGZdq{}}\PYG{l+s+s2}{V\PYGZus{}r}\PYG{l+s+s2}{\PYGZdq{}}\PYG{p}{,} \PYG{n}{Eq}\PYG{p}{(}\PYG{n}{V}\PYG{p}{,} \PYG{n}{radsimp}\PYG{p}{(}\PYG{n}{V\PYGZus{}r}\PYG{p}{)}\PYG{p}{)}\PYG{p}{)}
\PYG{n}{glue}\PYG{p}{(}\PYG{l+s+s2}{\PYGZdq{}}\PYG{l+s+s2}{V\PYGZus{}r\PYGZus{}f}\PYG{l+s+s2}{\PYGZdq{}}\PYG{p}{,} \PYG{n}{V\PYGZus{}r}\PYG{o}{.}\PYG{n}{evalf}\PYG{p}{(}\PYG{l+m+mi}{4}\PYG{p}{)}\PYG{p}{)}
\end{sphinxVerbatim}
\begin{equation*}
\begin{split}\displaystyle V = \frac{\sqrt{2} \left(48 + \sqrt{2845} + 6 \sqrt{145} + 4 \sqrt{673}\right)^{3}}{598752}\end{split}
\end{equation*}\begin{equation*}
\begin{split}\displaystyle 50.4\end{split}
\end{equation*}
\sphinxAtStartPar
This estimates a volume of \DUrole{pasted-inline}{\(\displaystyle V = \frac{\sqrt{2} \left(48 + \sqrt{2845} + 6 \sqrt{145} + 4 \sqrt{673}\right)^{3}}{598752}\)}, which is approximately \DUrole{pasted-text}{50.40} gallons.

\sphinxAtStartPar
This is close to the triangular prism estimate, so this is also a useful guage of the space.


\section{Smallest and Largest}
\label{\detokenize{tetrahedron:smallest-and-largest}}
\sphinxAtStartPar
Does it help to bracket this estimate with two other estimates, the least, \(l\), and the greatest, \(g\)? If we split these, it’s tolerably close to other estimates.

\begin{sphinxVerbatim}[commandchars=\\\{\}]
\PYG{n}{l} \PYG{o}{=} \PYG{n+nb}{min}\PYG{p}{(}\PYG{p}{[}\PYG{n}{a\PYGZus{}1}\PYG{p}{,} \PYG{n}{a\PYGZus{}2}\PYG{p}{,} \PYG{n}{a\PYGZus{}3}\PYG{p}{,} \PYG{n}{a\PYGZus{}4}\PYG{p}{,} \PYG{n}{a\PYGZus{}5}\PYG{p}{,} \PYG{n}{a\PYGZus{}6}\PYG{p}{]}\PYG{p}{)}
\PYG{n}{g} \PYG{o}{=} \PYG{n+nb}{max}\PYG{p}{(}\PYG{p}{[}\PYG{n}{a\PYGZus{}1}\PYG{p}{,} \PYG{n}{a\PYGZus{}2}\PYG{p}{,} \PYG{n}{a\PYGZus{}3}\PYG{p}{,} \PYG{n}{a\PYGZus{}4}\PYG{p}{,} \PYG{n}{a\PYGZus{}5}\PYG{p}{,} \PYG{n}{a\PYGZus{}6}\PYG{p}{]}\PYG{p}{)}
\PYG{n}{V\PYGZus{}l} \PYG{o}{=} \PYG{n}{l}\PYG{o}{*}\PYG{o}{*}\PYG{l+m+mi}{3} \PYG{o}{/} \PYG{p}{(}\PYG{l+m+mi}{6} \PYG{o}{*} \PYG{n}{sqrt}\PYG{p}{(}\PYG{l+m+mi}{2}\PYG{p}{)}\PYG{p}{)} \PYG{o}{/} \PYG{l+m+mi}{231}
\PYG{n}{V\PYGZus{}g} \PYG{o}{=} \PYG{n}{g}\PYG{o}{*}\PYG{o}{*}\PYG{l+m+mi}{3} \PYG{o}{/} \PYG{p}{(}\PYG{l+m+mi}{6} \PYG{o}{*} \PYG{n}{sqrt}\PYG{p}{(}\PYG{l+m+mi}{2}\PYG{p}{)}\PYG{p}{)} \PYG{o}{/} \PYG{l+m+mi}{231}
\end{sphinxVerbatim}

\begin{sphinxVerbatim}[commandchars=\\\{\}]
\PYG{n}{glue}\PYG{p}{(}\PYG{l+s+s2}{\PYGZdq{}}\PYG{l+s+s2}{l}\PYG{l+s+s2}{\PYGZdq{}}\PYG{p}{,} \PYG{n}{l}\PYG{p}{)}
\PYG{n}{glue}\PYG{p}{(}\PYG{l+s+s2}{\PYGZdq{}}\PYG{l+s+s2}{g}\PYG{l+s+s2}{\PYGZdq{}}\PYG{p}{,} \PYG{n}{g}\PYG{p}{)}
\PYG{n}{glue}\PYG{p}{(}\PYG{l+s+s2}{\PYGZdq{}}\PYG{l+s+s2}{V\PYGZus{}l}\PYG{l+s+s2}{\PYGZdq{}}\PYG{p}{,} \PYG{n}{V\PYGZus{}l}\PYG{p}{)}
\PYG{n}{glue}\PYG{p}{(}\PYG{l+s+s2}{\PYGZdq{}}\PYG{l+s+s2}{V\PYGZus{}l\PYGZus{}f}\PYG{l+s+s2}{\PYGZdq{}}\PYG{p}{,} \PYG{n}{V\PYGZus{}l}\PYG{o}{.}\PYG{n}{evalf}\PYG{p}{(}\PYG{l+m+mi}{4}\PYG{p}{)}\PYG{p}{)}
\PYG{n}{glue}\PYG{p}{(}\PYG{l+s+s2}{\PYGZdq{}}\PYG{l+s+s2}{V\PYGZus{}g}\PYG{l+s+s2}{\PYGZdq{}}\PYG{p}{,} \PYG{n}{V\PYGZus{}g}\PYG{p}{)}
\PYG{n}{glue}\PYG{p}{(}\PYG{l+s+s2}{\PYGZdq{}}\PYG{l+s+s2}{V\PYGZus{}g\PYGZus{}f}\PYG{l+s+s2}{\PYGZdq{}}\PYG{p}{,} \PYG{n}{V\PYGZus{}g}\PYG{o}{.}\PYG{n}{evalf}\PYG{p}{(}\PYG{l+m+mi}{4}\PYG{p}{)}\PYG{p}{)}
\PYG{n}{glue}\PYG{p}{(}\PYG{l+s+s2}{\PYGZdq{}}\PYG{l+s+s2}{V\PYGZus{}lg\PYGZus{}f}\PYG{l+s+s2}{\PYGZdq{}}\PYG{p}{,} \PYG{p}{(}\PYG{n}{V\PYGZus{}l}\PYG{o}{.}\PYG{n}{evalf}\PYG{p}{(}\PYG{l+m+mi}{4}\PYG{p}{)}\PYG{o}{+}\PYG{n}{V\PYGZus{}g}\PYG{o}{.}\PYG{n}{evalf}\PYG{p}{(}\PYG{l+m+mi}{4}\PYG{p}{)}\PYG{p}{)}\PYG{o}{/}\PYG{l+m+mi}{2}\PYG{p}{)}
\end{sphinxVerbatim}
\begin{equation*}
\begin{split}\displaystyle 3 \sqrt{145}\end{split}
\end{equation*}\begin{equation*}
\begin{split}\displaystyle \sqrt{2845}\end{split}
\end{equation*}\begin{equation*}
\begin{split}\displaystyle \frac{435 \sqrt{290}}{308}\end{split}
\end{equation*}\begin{equation*}
\begin{split}\displaystyle 24.05\end{split}
\end{equation*}\begin{equation*}
\begin{split}\displaystyle \frac{2845 \sqrt{5690}}{2772}\end{split}
\end{equation*}\begin{equation*}
\begin{split}\displaystyle 77.42\end{split}
\end{equation*}\begin{equation*}
\begin{split}\displaystyle 50.74\end{split}
\end{equation*}
\sphinxAtStartPar
This varies from \DUrole{pasted-inline}{\(\displaystyle \frac{435 \sqrt{290}}{308}\)} to \DUrole{pasted-inline}{\(\displaystyle \frac{2845 \sqrt{5690}}{2772}\)}.
These extemes are approximately \DUrole{pasted-text}{24.05} gallons to \DUrole{pasted-text}{77.42} gallons.
The midpoint, \DUrole{pasted-text}{50.74} gallons, also seems to agree with other estimates.


\section{Truncation}
\label{\detokenize{tetrahedron:truncation}}
\sphinxAtStartPar
There’s a small 9” tetrahedron we could truncate from this volume to improve accuracy. What’s its volume?

\begin{sphinxVerbatim}[commandchars=\\\{\}]
\PYG{n}{V\PYGZus{}t} \PYG{o}{=} \PYG{l+m+mi}{9}\PYG{o}{*}\PYG{o}{*}\PYG{l+m+mi}{3} \PYG{o}{/} \PYG{p}{(}\PYG{l+m+mi}{6} \PYG{o}{*} \PYG{n}{sqrt}\PYG{p}{(}\PYG{l+m+mi}{2}\PYG{p}{)}\PYG{p}{)} \PYG{o}{/} \PYG{l+m+mi}{231}
\end{sphinxVerbatim}

\begin{sphinxVerbatim}[commandchars=\\\{\}]
\PYG{n}{glue}\PYG{p}{(}\PYG{l+s+s2}{\PYGZdq{}}\PYG{l+s+s2}{V\PYGZus{}t}\PYG{l+s+s2}{\PYGZdq{}}\PYG{p}{,} \PYG{n}{V\PYGZus{}t}\PYG{p}{)}
\PYG{n}{glue}\PYG{p}{(}\PYG{l+s+s2}{\PYGZdq{}}\PYG{l+s+s2}{V\PYGZus{}t\PYGZus{}f}\PYG{l+s+s2}{\PYGZdq{}}\PYG{p}{,} \PYG{n}{V\PYGZus{}t}\PYG{o}{.}\PYG{n}{evalf}\PYG{p}{(}\PYG{l+m+mi}{4}\PYG{p}{)}\PYG{p}{)}
\end{sphinxVerbatim}
\begin{equation*}
\begin{split}\displaystyle \frac{81 \sqrt{2}}{308}\end{split}
\end{equation*}\begin{equation*}
\begin{split}\displaystyle 0.3719\end{split}
\end{equation*}
\sphinxAtStartPar
This tiny bit of space is \DUrole{pasted-inline}{\(\displaystyle \frac{81 \sqrt{2}}{308}\)}, which is approximately \DUrole{pasted-text}{0.37} gallons.
It’s negligible, however, being less than a gallon, and we can safely ignore it.

\sphinxAtStartPar
In the next section, we’ll apply some calculus to compute the volume as an infnite some of triangles, each a slightly different shape. We can model the taper from fore to aft, and use this to compute a more accurate volume.


\chapter{Tapered Triangular Prism}
\label{\detokenize{prism-irregular:tapered-triangular-prism}}\label{\detokenize{prism-irregular::doc}}
\sphinxAtStartPar
A regular prism is a solid with two triangular faces joined by three rectangular faces.

\sphinxAtStartPar
Referring back to the diagram, this is clearly not a good description of the space. The aft and and forward end are triangular, but the remaining three faces are not rectangles; the remaining faces are irregular quadrilaterals.

\sphinxAtStartPar
One end of the prism, the aft end, or “base”, is an isoscolese triangle, 27” by 48”.
The other end of the prism, the forward end, is an isoscolese triangle, 8” by 10½”.
The overall length is 46”.

\sphinxAtStartPar
We can use calculus to sum an infinte sequence of triangles from the small forward end to the large aft end. This will compute the volume of the prism.

\begin{sphinxVerbatim}[commandchars=\\\{\}]
\PYG{k+kn}{from} \PYG{n+nn}{myst\PYGZus{}nb} \PYG{k+kn}{import} \PYG{n}{glue}
\PYG{k+kn}{from} \PYG{n+nn}{sympy} \PYG{k+kn}{import} \PYG{o}{*}
\end{sphinxVerbatim}


\section{Approach}
\label{\detokenize{prism-irregular:approach}}
\sphinxAtStartPar
The volume, \(V\), of a regular triangular prism is the area of the triangle at either end, \(a\), integrated along the length of the prism, \(l\).
Since the area is fixed, \(a = \frac{h \times w}{2}\), we have this.
\begin{equation*}
\begin{split}
V_p = \int_0^l a \text{d}z = \int_{0}^{l} \frac{{h \times w}}{{2}} \text{d}z = \frac{h \times w \times l}{2}
\end{split}
\end{equation*}
\sphinxAtStartPar
In our case, we don’t have the same triangle at each end with a fixed area. Instead, we must define area as a function of the offset from the forward end of the tank, \(z\), \(a = A(z)\), leading to a slightly more complex integral.
\begin{equation*}
\begin{split}
V = \int_0^l A(z) \text{d}z
\end{split}
\end{equation*}
\sphinxAtStartPar
When we compute the area at any point along the fore\sphinxhyphen{}to\sphinxhyphen{}aft axis of the tank, we can accurately compute the volume within that infinite sequence of triangles.

\sphinxAtStartPar
As background, here’s the regular prism volume computation done with \sphinxcode{\sphinxupquote{sympy}}. This shows how we can translate the math to Python, and use this confirm our results.

\begin{sphinxVerbatim}[commandchars=\\\{\}]
\PYG{n}{h}\PYG{p}{,} \PYG{n}{w}\PYG{p}{,} \PYG{n}{l}\PYG{p}{,} \PYG{n}{z} \PYG{o}{=} \PYG{n}{symbols}\PYG{p}{(}\PYG{l+s+s2}{\PYGZdq{}}\PYG{l+s+s2}{h w l z}\PYG{l+s+s2}{\PYGZdq{}}\PYG{p}{)}

\PYG{n}{Integral}\PYG{p}{(}\PYG{n}{h}\PYG{o}{*}\PYG{n}{w}\PYG{o}{/}\PYG{l+m+mi}{2}\PYG{p}{,} \PYG{p}{(}\PYG{n}{z}\PYG{p}{,} \PYG{l+m+mi}{0}\PYG{p}{,} \PYG{n}{l}\PYG{p}{)}\PYG{p}{)}
\end{sphinxVerbatim}
\begin{equation*}
\begin{split}\displaystyle \int\limits_{0}^{l} \frac{h w}{2}\, dz\end{split}
\end{equation*}
\begin{sphinxVerbatim}[commandchars=\\\{\}]
\PYG{n}{Integral}\PYG{p}{(}\PYG{n}{h}\PYG{o}{*}\PYG{n}{w}\PYG{o}{/}\PYG{l+m+mi}{2}\PYG{p}{,} \PYG{p}{(}\PYG{n}{z}\PYG{p}{,} \PYG{l+m+mi}{0}\PYG{p}{,} \PYG{n}{l}\PYG{p}{)}\PYG{p}{)}\PYG{o}{.}\PYG{n}{doit}\PYG{p}{(}\PYG{p}{)}
\end{sphinxVerbatim}
\begin{equation*}
\begin{split}\displaystyle \frac{h l w}{2}\end{split}
\end{equation*}
\sphinxAtStartPar
The first step, then, is to compute the area of any triangle from the forward\sphinxhyphen{}most \(10 \tfrac{1}{2} \times 8\) to the aft\sphinxhyphen{}most \(48 \times 27\).


\section{Measurements}
\label{\detokenize{prism-irregular:measurements}}
\sphinxAtStartPar
Here are the essential measurements. We’ve defined these as a dictionary so we can substitute them into other equations. This allows us to change the measurements and get results.

\begin{sphinxVerbatim}[commandchars=\\\{\}]
\PYG{n}{var}\PYG{p}{(}\PYG{l+s+s2}{\PYGZdq{}}\PYG{l+s+s2}{h\PYGZus{}f, h\PYGZus{}a, w\PYGZus{}f, w\PYGZus{}a, l\PYGZus{}fa}\PYG{l+s+s2}{\PYGZdq{}}\PYG{p}{)}
\PYG{n}{measured} \PYG{o}{=} \PYG{p}{\PYGZob{}}
    \PYG{c+c1}{\PYGZsh{} Forward triangle, in inches}
    \PYG{l+s+s2}{\PYGZdq{}}\PYG{l+s+s2}{h\PYGZus{}f}\PYG{l+s+s2}{\PYGZdq{}}\PYG{p}{:} \PYG{l+m+mi}{8}\PYG{p}{,}
    \PYG{l+s+s2}{\PYGZdq{}}\PYG{l+s+s2}{w\PYGZus{}f}\PYG{l+s+s2}{\PYGZdq{}}\PYG{p}{:} \PYG{l+m+mi}{10} \PYG{o}{+} \PYG{n}{Rational}\PYG{p}{(}\PYG{l+m+mi}{1}\PYG{p}{,} \PYG{l+m+mi}{2}\PYG{p}{)}\PYG{p}{,}

    \PYG{c+c1}{\PYGZsh{} Aft triangle, in inches}
    \PYG{l+s+s2}{\PYGZdq{}}\PYG{l+s+s2}{h\PYGZus{}a}\PYG{l+s+s2}{\PYGZdq{}}\PYG{p}{:} \PYG{l+m+mi}{27}\PYG{p}{,}
    \PYG{l+s+s2}{\PYGZdq{}}\PYG{l+s+s2}{w\PYGZus{}a}\PYG{l+s+s2}{\PYGZdq{}}\PYG{p}{:} \PYG{l+m+mi}{48}\PYG{p}{,}

    \PYG{c+c1}{\PYGZsh{} Overall length from forward to aft, in inches.}
    \PYG{l+s+s2}{\PYGZdq{}}\PYG{l+s+s2}{l\PYGZus{}fa}\PYG{l+s+s2}{\PYGZdq{}}\PYG{p}{:} \PYG{l+m+mi}{46}\PYG{p}{,}
\PYG{p}{\PYGZcb{}}
\end{sphinxVerbatim}

\sphinxAtStartPar
Here are the essential width as a function of height, \(w_h\), and area based on height, \(A_h\), given a height of the triangle, \(h\).

\begin{sphinxVerbatim}[commandchars=\\\{\}]
\PYG{n}{h}\PYG{p}{,} \PYG{n}{w\PYGZus{}h}\PYG{p}{,} \PYG{n}{l}\PYG{p}{,} \PYG{n}{A}\PYG{p}{,} \PYG{n}{z}\PYG{p}{,} \PYG{n}{A\PYGZus{}h}\PYG{p}{,} \PYG{n}{A\PYGZus{}z} \PYG{o}{=} \PYG{n}{symbols}\PYG{p}{(}\PYG{l+s+s1}{\PYGZsq{}}\PYG{l+s+s1}{h w\PYGZus{}h l A z A\PYGZus{}h A\PYGZus{}z}\PYG{l+s+s1}{\PYGZsq{}}\PYG{p}{)}

\PYG{n}{glue}\PYG{p}{(}\PYG{l+s+s2}{\PYGZdq{}}\PYG{l+s+s2}{wfh}\PYG{l+s+s2}{\PYGZdq{}}\PYG{p}{,} \PYG{n}{Eq}\PYG{p}{(}\PYG{n}{w\PYGZus{}h}\PYG{p}{,} \PYG{n}{factor}\PYG{p}{(}\PYG{n}{w\PYGZus{}a}\PYG{o}{/}\PYG{n}{h\PYGZus{}a} \PYG{o}{*} \PYG{n}{h}\PYG{p}{)}\PYG{p}{,} \PYG{n}{evaluate}\PYG{o}{=}\PYG{k+kc}{False}\PYG{p}{)}\PYG{p}{)}
\PYG{n}{w\PYGZus{}h} \PYG{o}{=} \PYG{n}{w\PYGZus{}a}\PYG{o}{/}\PYG{n}{h\PYGZus{}a} \PYG{o}{*} \PYG{n}{h}

\PYG{n}{glue}\PYG{p}{(}\PYG{l+s+s2}{\PYGZdq{}}\PYG{l+s+s2}{area\PYGZus{}h}\PYG{l+s+s2}{\PYGZdq{}}\PYG{p}{,} \PYG{n}{Eq}\PYG{p}{(}\PYG{n}{A\PYGZus{}h}\PYG{p}{,} \PYG{n}{Rational}\PYG{p}{(}\PYG{l+m+mi}{1}\PYG{p}{,} \PYG{l+m+mi}{2}\PYG{p}{)} \PYG{o}{*} \PYG{n}{h} \PYG{o}{*} \PYG{n}{w\PYGZus{}h}\PYG{p}{)}\PYG{p}{)}
\end{sphinxVerbatim}
\begin{equation*}
\begin{split}\displaystyle w_{h} = \frac{h w_{a}}{h_{a}}\end{split}
\end{equation*}\begin{equation*}
\begin{split}\displaystyle A_{h} = \frac{h^{2} w_{a}}{2 h_{a}}\end{split}
\end{equation*}
\sphinxAtStartPar
If we assume the fore and aft triangles are congruent, then the width, \(w\), of the triangular face of the prism is a function of height, \(h\).
\begin{equation}\label{equation:index:width_from_height}
\begin{split}\displaystyle w_{h} = \frac{h w_{a}}{h_{a}}\end{split}
\end{equation}
\sphinxAtStartPar
Area, \(A = \frac{1}{2} \times w \times h\), can then become a function of height, \(h\).
\begin{equation}\label{equation:index:area_from_height}
\begin{split}\displaystyle A_{h} = \frac{h^{2} w_{a}}{2 h_{a}}\end{split}
\end{equation}
\sphinxAtStartPar
Height, \(h\), depends on the distance along the z\sphinxhyphen{}distance, meadured from the front of the tank.
The \(z=0\) position is the forward edge, with a measured height of \(h_f\).
The \(z=l_{fa}\) position is length from the forward edge to the aft edge; this has a measured height of \(h_a\).
\begin{equation*}
\begin{split}
h(z) = \frac{\Delta h}{\Delta z} \times z + h_f  = \frac{h_a-h_f}{l_{fa}} \times z + h_f
\end{split}
\end{equation*}
\sphinxAtStartPar
We’re assuming each of the two ends of the prism are congruent. This means we’re imposing the aft width and height on the tank as a whole.

\sphinxAtStartPar
Is this a safe assumption? Spoiler alert: it isn’t.


\section{Challenging the Assumption}
\label{\detokenize{prism-irregular:challenging-the-assumption}}
\sphinxAtStartPar
We’ve assumed the shape of the forward and aft end of the tank are congruent triangles.
We’re treating this as a triangular section that grows in size from a small 8” by 10½” triangle
to a large 27” by 48” triangle.

\sphinxAtStartPar
Since \(\frac{8}{10 \tfrac{1}{2}} \neq \frac{27}{48}\), we can see the two triangles aren’t congruent. This simplifies to \(\frac{16}{21} \neq \frac{9}{16}\).

\sphinxAtStartPar
Here are the two triangle area computations based on height. \(A_a(h)\) uses the aft height\sphinxhyphen{}to\sphinxhyphen{}width ratio,
\(A_f(h)\) uses the forward height\sphinxhyphen{}to\sphinxhyphen{}width ratio. The \(A(h)\) computations are quite different.

\begin{sphinxVerbatim}[commandchars=\\\{\}]
\PYG{n}{w\PYGZus{}a\PYGZus{}h} \PYG{o}{=} \PYG{n}{w\PYGZus{}a}\PYG{o}{/}\PYG{n}{h\PYGZus{}a} \PYG{o}{*} \PYG{n}{h}
\PYG{n}{w\PYGZus{}f\PYGZus{}h} \PYG{o}{=} \PYG{n}{w\PYGZus{}f}\PYG{o}{/}\PYG{n}{h\PYGZus{}f} \PYG{o}{*} \PYG{n}{h}

\PYG{n}{A\PYGZus{}a\PYGZus{}h} \PYG{o}{=} \PYG{n}{Rational}\PYG{p}{(}\PYG{l+m+mi}{1}\PYG{p}{,} \PYG{l+m+mi}{2}\PYG{p}{)} \PYG{o}{*} \PYG{n}{h} \PYG{o}{*} \PYG{n}{w\PYGZus{}a\PYGZus{}h}
\PYG{n}{A\PYGZus{}f\PYGZus{}h} \PYG{o}{=} \PYG{n}{Rational}\PYG{p}{(}\PYG{l+m+mi}{1}\PYG{p}{,} \PYG{l+m+mi}{2}\PYG{p}{)} \PYG{o}{*} \PYG{n}{h} \PYG{o}{*} \PYG{n}{w\PYGZus{}f\PYGZus{}h}
\end{sphinxVerbatim}

\begin{sphinxVerbatim}[commandchars=\\\{\}]
\PYG{n}{A\PYGZus{}a\PYGZus{}h}\PYG{o}{.}\PYG{n}{subs}\PYG{p}{(}\PYG{n}{measured}\PYG{p}{)}
\end{sphinxVerbatim}
\begin{equation*}
\begin{split}\displaystyle \frac{8 h^{2}}{9}\end{split}
\end{equation*}
\begin{sphinxVerbatim}[commandchars=\\\{\}]
\PYG{n}{A\PYGZus{}a\PYGZus{}h}\PYG{o}{.}\PYG{n}{subs}\PYG{p}{(}\PYG{n}{measured}\PYG{p}{)}\PYG{o}{.}\PYG{n}{evalf}\PYG{p}{(}\PYG{l+m+mi}{3}\PYG{p}{)}
\end{sphinxVerbatim}
\begin{equation*}
\begin{split}\displaystyle 0.889 h^{2}\end{split}
\end{equation*}
\begin{sphinxVerbatim}[commandchars=\\\{\}]
\PYG{n}{A\PYGZus{}f\PYGZus{}h}\PYG{o}{.}\PYG{n}{subs}\PYG{p}{(}\PYG{n}{measured}\PYG{p}{)}
\end{sphinxVerbatim}
\begin{equation*}
\begin{split}\displaystyle \frac{21 h^{2}}{32}\end{split}
\end{equation*}
\begin{sphinxVerbatim}[commandchars=\\\{\}]
\PYG{n}{A\PYGZus{}f\PYGZus{}h}\PYG{o}{.}\PYG{n}{subs}\PYG{p}{(}\PYG{n}{measured}\PYG{p}{)}\PYG{o}{.}\PYG{n}{evalf}\PYG{p}{(}\PYG{l+m+mi}{3}\PYG{p}{)}
\end{sphinxVerbatim}
\begin{equation*}
\begin{split}\displaystyle 0.656 h^{2}\end{split}
\end{equation*}
\sphinxAtStartPar
This means a simple \(A(h)\) computation using the height, \(h\), to compute the area isn’t going to be very useful. We have two choices:
\begin{itemize}
\item {} 
\sphinxAtStartPar
Define area, \(A(z)\) based on independent  \(h(z)\) and \(w(z)\). We can use \(A(z) = \frac{h(z) \times w(z)}{2}\).

\item {} 
\sphinxAtStartPar
Use a midpoint ratio of width to height, \(r_m\), to define area. If \(w = r_m \times h(z)\), then \(A(z) = \frac{h(z) \times w}{2} = \frac{h(z)^2 \times r_m}{2}\).

\end{itemize}

\sphinxAtStartPar
Using a midpoint ratio is slightly simpler, but suffers from a problem of being inaccurate. The difference between \(0.889h^2\) and \(0.656h^2\) for small values of \(h\) will be significant.

\sphinxAtStartPar
We need to compute the area using independent \(h(z)\) and \(w(z)\) functions.


\section{Computing the Area}
\label{\detokenize{prism-irregular:computing-the-area}}
\sphinxAtStartPar
We can use independent \(h(z)\) and \(w(z)\) functions to compute the overall area of each triangular section of the tank.

\sphinxAtStartPar
We’ll assume these are linear functions of the form \(y = mx + b\). The slope, \(m\), for height is \(\frac{\Delta h}{\Delta z}\), and the intercept, \(b\), is the height at the forward end, \(h_f\). The width equation is similar.

\sphinxAtStartPar
This leads to two equations for \(h_z\) and \(w_z\), which are functions of the distance from the forward end, \(z\).

\begin{sphinxVerbatim}[commandchars=\\\{\}]
\PYG{n}{h\PYGZus{}z} \PYG{o}{=} \PYG{p}{(}\PYG{n}{h\PYGZus{}a} \PYG{o}{\PYGZhy{}} \PYG{n}{h\PYGZus{}f}\PYG{p}{)} \PYG{o}{/} \PYG{n}{l\PYGZus{}fa} \PYG{o}{*} \PYG{n}{z} \PYG{o}{+} \PYG{n}{h\PYGZus{}f}
\PYG{n}{glue}\PYG{p}{(}\PYG{l+s+s2}{\PYGZdq{}}\PYG{l+s+s2}{hfz}\PYG{l+s+s2}{\PYGZdq{}}\PYG{p}{,} \PYG{n}{h\PYGZus{}z}\PYG{p}{)}
\PYG{n}{w\PYGZus{}z} \PYG{o}{=} \PYG{p}{(}\PYG{n}{w\PYGZus{}a} \PYG{o}{\PYGZhy{}} \PYG{n}{w\PYGZus{}f}\PYG{p}{)} \PYG{o}{/} \PYG{n}{l\PYGZus{}fa} \PYG{o}{*} \PYG{n}{z} \PYG{o}{+} \PYG{n}{w\PYGZus{}f}
\PYG{n}{glue}\PYG{p}{(}\PYG{l+s+s2}{\PYGZdq{}}\PYG{l+s+s2}{wfz}\PYG{l+s+s2}{\PYGZdq{}}\PYG{p}{,} \PYG{n}{w\PYGZus{}z}\PYG{p}{)}
\end{sphinxVerbatim}
\begin{equation*}
\begin{split}\displaystyle h_{f} + \frac{z \left(h_{a} - h_{f}\right)}{l_{fa}}\end{split}
\end{equation*}\begin{equation*}
\begin{split}\displaystyle w_{f} + \frac{z \left(w_{a} - w_{f}\right)}{l_{fa}}\end{split}
\end{equation*}
\sphinxAtStartPar
The height, \(h(z) =\) \DUrole{pasted-inline}{\(\displaystyle h_{f} + \frac{z \left(h_{a} - h_{f}\right)}{l_{fa}}\)}. The width, \(w(z) =\) \DUrole{pasted-inline}{\(\displaystyle w_{f} + \frac{z \left(w_{a} - w_{f}\right)}{l_{fa}}\)}.

\begin{sphinxVerbatim}[commandchars=\\\{\}]
\PYG{n}{A\PYGZus{}z} \PYG{o}{=} \PYG{n}{factor}\PYG{p}{(}\PYG{n}{expand}\PYG{p}{(}\PYG{n}{Rational}\PYG{p}{(}\PYG{l+m+mi}{1}\PYG{p}{,} \PYG{l+m+mi}{2}\PYG{p}{)} \PYG{o}{*} \PYG{n}{h\PYGZus{}z} \PYG{o}{*} \PYG{n}{w\PYGZus{}z}\PYG{p}{)}\PYG{p}{)}
\PYG{n}{glue}\PYG{p}{(}\PYG{l+s+s2}{\PYGZdq{}}\PYG{l+s+s2}{area\PYGZus{}z}\PYG{l+s+s2}{\PYGZdq{}}\PYG{p}{,} \PYG{n}{A\PYGZus{}z}\PYG{p}{)}
\end{sphinxVerbatim}
\begin{equation*}
\begin{split}\displaystyle \frac{\left(h_{a} z + h_{f} l_{fa} - h_{f} z\right) \left(l_{fa} w_{f} + w_{a} z - w_{f} z\right)}{2 l_{fa}^{2}}\end{split}
\end{equation*}
\sphinxAtStartPar
From \(h(z)\) and \(w(z)\), we can compute the area, \(A_z\), as a function of the distance along the Z axis, \(A(z) = \frac{1}{2} h(z) w(z) = \) \DUrole{pasted-inline}{\(\displaystyle \frac{\left(h_{a} z + h_{f} l_{fa} - h_{f} z\right) \left(l_{fa} w_{f} + w_{a} z - w_{f} z\right)}{2 l_{fa}^{2}}\)}.

\sphinxAtStartPar
This is a bit bulky. We can try to simplify it. First, however, we need to test it to be sure it produces proper area values.

\sphinxAtStartPar
We can evaluate the \(h(z)\) and \(w(z)\) functions at \(z=0\) and \(z=l_{fa}\) to be sure we’ve got them right.  We expect \(h(0) = h_f\), \(h(l_{fa}) = h_a\), \(w(0) = w_f\), and \(w(l_{fa}) = w_a\). We can also substitute the actual measurements to compute values for the fore and aft triangles to be sure they match the original measurements.

\begin{sphinxVerbatim}[commandchars=\\\{\}]
\PYG{n}{h\PYGZus{}z}\PYG{o}{.}\PYG{n}{subs}\PYG{p}{(}\PYG{p}{\PYGZob{}}\PYG{n}{z}\PYG{p}{:} \PYG{l+m+mi}{0}\PYG{p}{\PYGZcb{}}\PYG{p}{)}\PYG{o}{.}\PYG{n}{evalf}\PYG{p}{(}\PYG{p}{)}
\end{sphinxVerbatim}
\begin{equation*}
\begin{split}\displaystyle h_{f}\end{split}
\end{equation*}
\begin{sphinxVerbatim}[commandchars=\\\{\}]
\PYG{n}{w\PYGZus{}z}\PYG{o}{.}\PYG{n}{subs}\PYG{p}{(}\PYG{p}{\PYGZob{}}\PYG{n}{z}\PYG{p}{:} \PYG{l+m+mi}{0}\PYG{p}{\PYGZcb{}}\PYG{p}{)}\PYG{o}{.}\PYG{n}{evalf}\PYG{p}{(}\PYG{p}{)}
\end{sphinxVerbatim}
\begin{equation*}
\begin{split}\displaystyle w_{f}\end{split}
\end{equation*}
\begin{sphinxVerbatim}[commandchars=\\\{\}]
\PYG{n}{h\PYGZus{}z}\PYG{o}{.}\PYG{n}{subs}\PYG{p}{(}\PYG{p}{\PYGZob{}}\PYG{n}{z}\PYG{p}{:} \PYG{n}{l\PYGZus{}fa}\PYG{p}{\PYGZcb{}}\PYG{p}{)}\PYG{o}{.}\PYG{n}{evalf}\PYG{p}{(}\PYG{p}{)}
\end{sphinxVerbatim}
\begin{equation*}
\begin{split}\displaystyle h_{a}\end{split}
\end{equation*}
\begin{sphinxVerbatim}[commandchars=\\\{\}]
\PYG{n}{w\PYGZus{}z}\PYG{o}{.}\PYG{n}{subs}\PYG{p}{(}\PYG{p}{\PYGZob{}}\PYG{n}{z}\PYG{p}{:} \PYG{n}{l\PYGZus{}fa}\PYG{p}{\PYGZcb{}}\PYG{p}{)}\PYG{o}{.}\PYG{n}{evalf}\PYG{p}{(}\PYG{p}{)}
\end{sphinxVerbatim}
\begin{equation*}
\begin{split}\displaystyle w_{a}\end{split}
\end{equation*}
\begin{sphinxVerbatim}[commandchars=\\\{\}]
\PYG{n}{glue}\PYG{p}{(}\PYG{l+s+s2}{\PYGZdq{}}\PYG{l+s+s2}{h\PYGZus{}f}\PYG{l+s+s2}{\PYGZdq{}}\PYG{p}{,} \PYG{n}{h\PYGZus{}z}\PYG{o}{.}\PYG{n}{subs}\PYG{p}{(}\PYG{n}{measured}\PYG{p}{)}\PYG{o}{.}\PYG{n}{subs}\PYG{p}{(}\PYG{p}{\PYGZob{}}\PYG{n}{z}\PYG{p}{:} \PYG{l+m+mi}{0}\PYG{p}{\PYGZcb{}}\PYG{p}{)}\PYG{o}{.}\PYG{n}{evalf}\PYG{p}{(}\PYG{p}{)}\PYG{p}{)}
\end{sphinxVerbatim}
\begin{equation*}
\begin{split}\displaystyle 8.0\end{split}
\end{equation*}
\begin{sphinxVerbatim}[commandchars=\\\{\}]
\PYG{n}{glue}\PYG{p}{(}\PYG{l+s+s2}{\PYGZdq{}}\PYG{l+s+s2}{w\PYGZus{}f}\PYG{l+s+s2}{\PYGZdq{}}\PYG{p}{,} \PYG{n}{w\PYGZus{}z}\PYG{o}{.}\PYG{n}{subs}\PYG{p}{(}\PYG{n}{measured}\PYG{p}{)}\PYG{o}{.}\PYG{n}{subs}\PYG{p}{(}\PYG{p}{\PYGZob{}}\PYG{n}{z}\PYG{p}{:} \PYG{l+m+mi}{0}\PYG{p}{\PYGZcb{}}\PYG{p}{)}\PYG{o}{.}\PYG{n}{evalf}\PYG{p}{(}\PYG{p}{)}\PYG{p}{)}
\end{sphinxVerbatim}
\begin{equation*}
\begin{split}\displaystyle 10.5\end{split}
\end{equation*}
\begin{sphinxVerbatim}[commandchars=\\\{\}]
\PYG{n}{glue}\PYG{p}{(}\PYG{l+s+s2}{\PYGZdq{}}\PYG{l+s+s2}{h\PYGZus{}a}\PYG{l+s+s2}{\PYGZdq{}}\PYG{p}{,} \PYG{n}{h\PYGZus{}z}\PYG{o}{.}\PYG{n}{subs}\PYG{p}{(}\PYG{n}{measured}\PYG{p}{)}\PYG{o}{.}\PYG{n}{subs}\PYG{p}{(}\PYG{p}{\PYGZob{}}\PYG{n}{z}\PYG{p}{:} \PYG{n}{measured}\PYG{p}{[}\PYG{l+s+s1}{\PYGZsq{}}\PYG{l+s+s1}{l\PYGZus{}fa}\PYG{l+s+s1}{\PYGZsq{}}\PYG{p}{]}\PYG{p}{\PYGZcb{}}\PYG{p}{)}\PYG{o}{.}\PYG{n}{evalf}\PYG{p}{(}\PYG{p}{)}\PYG{p}{)}
\end{sphinxVerbatim}
\begin{equation*}
\begin{split}\displaystyle 27.0\end{split}
\end{equation*}
\begin{sphinxVerbatim}[commandchars=\\\{\}]
\PYG{n}{glue}\PYG{p}{(}\PYG{l+s+s2}{\PYGZdq{}}\PYG{l+s+s2}{w\PYGZus{}a}\PYG{l+s+s2}{\PYGZdq{}}\PYG{p}{,} \PYG{n}{w\PYGZus{}z}\PYG{o}{.}\PYG{n}{subs}\PYG{p}{(}\PYG{n}{measured}\PYG{p}{)}\PYG{o}{.}\PYG{n}{subs}\PYG{p}{(}\PYG{p}{\PYGZob{}}\PYG{n}{z}\PYG{p}{:} \PYG{n}{measured}\PYG{p}{[}\PYG{l+s+s1}{\PYGZsq{}}\PYG{l+s+s1}{l\PYGZus{}fa}\PYG{l+s+s1}{\PYGZsq{}}\PYG{p}{]}\PYG{p}{\PYGZcb{}}\PYG{p}{)}\PYG{o}{.}\PYG{n}{evalf}\PYG{p}{(}\PYG{p}{)}\PYG{p}{)}
\end{sphinxVerbatim}
\begin{equation*}
\begin{split}\displaystyle 48.0\end{split}
\end{equation*}
\sphinxAtStartPar
To confirm that we’ve done this right so far, let’s check the model against reality.

\sphinxAtStartPar
At the forward end of the tank, this model predicts a triangle \DUrole{pasted-text}{10.5} across the top,
with a height of \DUrole{pasted-text}{8.0}. This matches the \(10.5 \times 8\) actual.

\sphinxAtStartPar
At the aft end of the tank, this model predicts a triangle \DUrole{pasted-text}{48.0} across the top,
with a height of \DUrole{pasted-text}{27.0}. This matches the \(48 \times 27\) actual, also.

\sphinxAtStartPar
Now that we can compute the shape of the triangle at each end of the space, we can compute the area, \(A(z) = \frac{h(z) w(z)}{2}\). From this, we can then compute the volume.


\section{Volume based on overall length}
\label{\detokenize{prism-irregular:volume-based-on-overall-length}}
\sphinxAtStartPar
The volume is the integral of the areas, \(A(z)\) where \(z\) varies from zero to the length of the prism, \(l_fa\).
\begin{equation*}
\begin{split}
V = \int_{0}^{l_{fa}} A(z) \text{d}z
\end{split}
\end{equation*}
\sphinxAtStartPar
For a regular prism this is the \(V_p = \frac{h l w}{2}\) formula. Our area is not simply \(\frac{hw}{2}\), it’s \(A(z) = \) \DUrole{pasted-inline}{\(\displaystyle \frac{\left(h_{a} z + h_{f} l_{fa} - h_{f} z\right) \left(l_{fa} w_{f} + w_{a} z - w_{f} z\right)}{2 l_{fa}^{2}}\)}.

\begin{sphinxVerbatim}[commandchars=\\\{\}]
\PYG{n}{var}\PYG{p}{(}\PYG{l+s+s2}{\PYGZdq{}}\PYG{l+s+s2}{V}\PYG{l+s+s2}{\PYGZdq{}}\PYG{p}{)}
\PYG{n}{glue}\PYG{p}{(}\PYG{l+s+s2}{\PYGZdq{}}\PYG{l+s+s2}{V}\PYG{l+s+s2}{\PYGZdq{}}\PYG{p}{,} \PYG{n}{Eq}\PYG{p}{(}\PYG{n}{V}\PYG{p}{,} \PYG{n}{Integral}\PYG{p}{(}\PYG{n}{A\PYGZus{}z}\PYG{p}{,} \PYG{p}{(}\PYG{n}{z}\PYG{p}{,} \PYG{l+m+mi}{0}\PYG{p}{,} \PYG{n}{l\PYGZus{}fa}\PYG{p}{)}\PYG{p}{)}\PYG{p}{)}\PYG{p}{)}
\end{sphinxVerbatim}
\begin{equation*}
\begin{split}\displaystyle V = \int\limits_{0}^{l_{fa}} \frac{\left(h_{a} z + h_{f} l_{fa} - h_{f} z\right) \left(l_{fa} w_{f} + w_{a} z - w_{f} z\right)}{2 l_{fa}^{2}}\, dz\end{split}
\end{equation*}
\sphinxAtStartPar
The volume is computed with
\begin{equation}\label{equation:index:volume}
\begin{split}\displaystyle V = \int\limits_{0}^{l_{fa}} \frac{\left(h_{a} z + h_{f} l_{fa} - h_{f} z\right) \left(l_{fa} w_{f} + w_{a} z - w_{f} z\right)}{2 l_{fa}^{2}}\, dz\end{split}
\end{equation}
\sphinxAtStartPar
We can substitute our measurements to get the volume. We’ll apply the magical 231 cubic inch per gallon factor to get the volume in gallons of fresh water.

\begin{sphinxVerbatim}[commandchars=\\\{\}]
\PYG{n}{V} \PYG{o}{=} \PYG{n}{Integral}\PYG{p}{(}\PYG{n}{A\PYGZus{}z}\PYG{o}{.}\PYG{n}{subs}\PYG{p}{(}\PYG{n}{measured}\PYG{p}{)}\PYG{p}{,} \PYG{p}{(}\PYG{n}{z}\PYG{p}{,} \PYG{l+m+mi}{0}\PYG{p}{,} \PYG{n}{measured}\PYG{p}{[}\PYG{l+s+s1}{\PYGZsq{}}\PYG{l+s+s1}{l\PYGZus{}fa}\PYG{l+s+s1}{\PYGZsq{}}\PYG{p}{]}\PYG{p}{)}\PYG{p}{)}
\PYG{n}{V\PYGZus{}r} \PYG{o}{=} \PYG{p}{(}\PYG{n}{V}\PYG{o}{.}\PYG{n}{doit}\PYG{p}{(}\PYG{p}{)}\PYG{o}{/}\PYG{l+m+mi}{231}\PYG{p}{)}\PYG{o}{.}\PYG{n}{limit\PYGZus{}denominator}\PYG{p}{(}\PYG{l+m+mi}{100}\PYG{p}{)}
\PYG{l+s+sa}{f}\PYG{l+s+s2}{\PYGZdq{}}\PYG{l+s+si}{\PYGZob{}}\PYG{n}{floor}\PYG{p}{(}\PYG{n}{V\PYGZus{}r}\PYG{p}{)}\PYG{l+s+si}{\PYGZcb{}}\PYG{l+s+s2}{ }\PYG{l+s+si}{\PYGZob{}}\PYG{n}{frac}\PYG{p}{(}\PYG{n}{V\PYGZus{}r}\PYG{p}{)}\PYG{l+s+si}{\PYGZcb{}}\PYG{l+s+s2}{ gallons}\PYG{l+s+s2}{\PYGZdq{}}
\end{sphinxVerbatim}

\begin{sphinxVerbatim}[commandchars=\\\{\}]
\PYGZsq{}56 79/90 gallons\PYGZsq{}
\end{sphinxVerbatim}

\begin{sphinxVerbatim}[commandchars=\\\{\}]
\PYG{n}{V\PYGZus{}r}\PYG{o}{.}\PYG{n}{evalf}\PYG{p}{(}\PYG{l+m+mi}{3}\PYG{p}{)}
\end{sphinxVerbatim}
\begin{equation*}
\begin{split}\displaystyle 56.9\end{split}
\end{equation*}

\section{Simplified form}
\label{\detokenize{prism-irregular:simplified-form}}
\sphinxAtStartPar
We can create decimal approximations for the fractions, and work with a direct computation that avoids integration. It’s not clear that this is simpler. The generic \sphinxcode{\sphinxupquote{simplify()}} is a poor choice.

\begin{sphinxVerbatim}[commandchars=\\\{\}]
\PYG{n}{var}\PYG{p}{(}\PYG{l+s+s2}{\PYGZdq{}}\PYG{l+s+s2}{l}\PYG{l+s+s2}{\PYGZdq{}}\PYG{p}{)}
\PYG{n}{simplify}\PYG{p}{(}\PYG{n}{Integral}\PYG{p}{(}\PYG{p}{(}\PYG{n}{A\PYGZus{}z}\PYG{o}{/}\PYG{l+m+mi}{231}\PYG{p}{)}\PYG{o}{.}\PYG{n}{evalf}\PYG{p}{(}\PYG{l+m+mi}{3}\PYG{p}{)}\PYG{p}{,} \PYG{p}{(}\PYG{n}{z}\PYG{p}{,} \PYG{l+m+mi}{0}\PYG{p}{,} \PYG{n}{l\PYGZus{}fa}\PYG{p}{)}\PYG{p}{)}\PYG{p}{)}
\end{sphinxVerbatim}
\begin{equation*}
\begin{split}\displaystyle \frac{0.00216 \left(h_{f} l_{fa}^{3} w_{f} + l_{fa}^{3} \left(\frac{h_{a} w_{a}}{3} - \frac{h_{a} w_{f}}{3} - \frac{h_{f} w_{a}}{3} + \frac{h_{f} w_{f}}{3}\right) + l_{fa}^{2} \left(\frac{h_{a} l_{fa} w_{f}}{2} + \frac{h_{f} l_{fa} w_{a}}{2} - h_{f} l_{fa} w_{f}\right)\right)}{l_{fa}^{2}}\end{split}
\end{equation*}
\sphinxAtStartPar
This variation collects the various factors together, giving a closed form that’s kind of workable. It involves terms based on \(l_{fa}\), \(l_{fa}^2\), and \(l_{fa}^3\) which seems about right. This has a single constant term out front for the conversion from cubic inches to gallons.

\sphinxAtStartPar
(Using cubic centimeters and liters would avoid the magical 231 cubic inches per gallon.)

\sphinxAtStartPar
We can try and factor the polynomial , which leads to a much simpler\sphinxhyphen{}looking computation of volume.

\begin{sphinxVerbatim}[commandchars=\\\{\}]
\PYG{n}{V\PYGZus{}c} \PYG{o}{=} \PYG{n}{factor}\PYG{p}{(}\PYG{n}{simplify}\PYG{p}{(}\PYG{n}{Integral}\PYG{p}{(}\PYG{p}{(}\PYG{n}{A\PYGZus{}z}\PYG{o}{/}\PYG{l+m+mi}{231}\PYG{p}{)}\PYG{o}{.}\PYG{n}{evalf}\PYG{p}{(}\PYG{l+m+mi}{3}\PYG{p}{)}\PYG{p}{,} \PYG{p}{(}\PYG{n}{z}\PYG{p}{,} \PYG{l+m+mi}{0}\PYG{p}{,} \PYG{n}{l\PYGZus{}fa}\PYG{p}{)}\PYG{p}{)}\PYG{p}{)}\PYG{p}{)}
\PYG{n}{V\PYGZus{}c}
\end{sphinxVerbatim}
\begin{equation*}
\begin{split}\displaystyle 0.000361 l_{fa} \left(2 h_{a} w_{a} + h_{a} w_{f} + h_{f} w_{a} + 2 h_{f} w_{f}\right)\end{split}
\end{equation*}
\sphinxAtStartPar
This seems to be an elegant closed\sphinxhyphen{}form equatio for computing volume from the given measurements. We can recompute the volume as our measurements improve.

\begin{sphinxVerbatim}[commandchars=\\\{\}]
\PYG{n}{V\PYGZus{}c}\PYG{o}{.}\PYG{n}{subs}\PYG{p}{(}\PYG{n}{measured}\PYG{p}{)}
\end{sphinxVerbatim}
\begin{equation*}
\begin{split}\displaystyle 56.9\end{split}
\end{equation*}

\section{Matrices}
\label{\detokenize{prism-irregular:matrices}}
\sphinxAtStartPar
Note that in the closed form volume equation, each term has some combination of \(h_a\), \(w_a\), \(h_f\), and \(w_f\), and unavoidable source of complexity. This “sum\sphinxhyphen{}of\sphinxhyphen{}combinations” suggests there may is a matrix expression to summarize this complexity.

\sphinxAtStartPar
The following nonsense shows that we can reproduce the volume formula. The following nonsense lacks a clear interpretation. Because it happens to work, it’s likely related to the proper scalar triple product (or box product).

\begin{sphinxVerbatim}[commandchars=\\\{\}]
\PYG{n}{M\PYGZus{}h} \PYG{o}{=} \PYG{n}{Matrix}\PYG{p}{(}\PYG{p}{[}\PYG{n}{h\PYGZus{}a}\PYG{p}{,} \PYG{n}{h\PYGZus{}f}\PYG{p}{]}\PYG{p}{)}
\PYG{n}{M\PYGZus{}w} \PYG{o}{=} \PYG{n}{Matrix}\PYG{p}{(}\PYG{p}{[}\PYG{n}{w\PYGZus{}a}\PYG{p}{,} \PYG{n}{w\PYGZus{}f}\PYG{p}{]}\PYG{p}{)}
\end{sphinxVerbatim}

\begin{sphinxVerbatim}[commandchars=\\\{\}]
\PYG{n}{V\PYGZus{}m} \PYG{o}{=} \PYG{n}{l\PYGZus{}fa}\PYG{o}{*}\PYG{p}{(}\PYG{n}{M\PYGZus{}h}\PYG{o}{*}\PYG{n}{M\PYGZus{}w}\PYG{o}{.}\PYG{n}{transpose}\PYG{p}{(}\PYG{p}{)}\PYG{p}{)}\PYG{o}{.}\PYG{n}{vec}\PYG{p}{(}\PYG{p}{)}\PYG{o}{.}\PYG{n}{dot}\PYG{p}{(}\PYG{n}{Matrix}\PYG{p}{(}\PYG{p}{[}\PYG{n}{S}\PYG{p}{(}\PYG{l+m+mi}{1}\PYG{p}{)}\PYG{o}{/}\PYG{l+m+mi}{3}\PYG{p}{,} \PYG{n}{S}\PYG{p}{(}\PYG{l+m+mi}{1}\PYG{p}{)}\PYG{o}{/}\PYG{l+m+mi}{6}\PYG{p}{,} \PYG{n}{S}\PYG{p}{(}\PYG{l+m+mi}{1}\PYG{p}{)}\PYG{o}{/}\PYG{l+m+mi}{6}\PYG{p}{,} \PYG{n}{S}\PYG{p}{(}\PYG{l+m+mi}{1}\PYG{p}{)}\PYG{o}{/}\PYG{l+m+mi}{3}\PYG{p}{]}\PYG{p}{)}\PYG{p}{)}\PYG{o}{/}\PYG{p}{(}\PYG{l+m+mi}{2}\PYG{o}{*}\PYG{l+m+mi}{231}\PYG{p}{)}
\PYG{n}{V\PYGZus{}m}
\end{sphinxVerbatim}
\begin{equation*}
\begin{split}\displaystyle \frac{l_{fa} \left(\frac{h_{a} w_{a}}{3} + \frac{h_{a} w_{f}}{6} + \frac{h_{f} w_{a}}{6} + \frac{h_{f} w_{f}}{3}\right)}{462}\end{split}
\end{equation*}
\begin{sphinxVerbatim}[commandchars=\\\{\}]
\PYG{n}{V\PYGZus{}m}\PYG{o}{.}\PYG{n}{subs}\PYG{p}{(}\PYG{n}{measured}\PYG{p}{)}\PYG{o}{.}\PYG{n}{evalf}\PYG{p}{(}\PYG{l+m+mi}{3}\PYG{p}{)}
\end{sphinxVerbatim}
\begin{equation*}
\begin{split}\displaystyle 56.9\end{split}
\end{equation*}
\sphinxAtStartPar
We’ve left this in as a placeholder for future learning. This seems to be part of the parallelepiped dot product computation.


\chapter{Summary and Conclusion}
\label{\detokenize{conclusion:summary-and-conclusion}}\label{\detokenize{conclusion::doc}}
\sphinxAtStartPar
We’ve looked at a number of alternatives for estimating the volume of the V\sphinxhyphen{}berth fuel tank.
Each of these uses some slight simplifications of the geometry to make the math a little simpler.
\begin{itemize}
\item {} 
\sphinxAtStartPar
\sphinxstylestrong{Regular Triangular Prism}. We picked a mid\sphinxhyphen{}point between the two triangular faces and used this to estimate the overall size. The resulting volume was 50.97 gallons.

\item {} 
\sphinxAtStartPar
\sphinxstylestrong{Regular Tetrahedron}. We used two approaches here.
\begin{itemize}
\item {} 
\sphinxAtStartPar
We picked a mid\sphinxhyphen{}point among the various edges and used this to estimate the size. The resulting volume was 50.4 gallons.

\item {} 
\sphinxAtStartPar
We used the largest and smallest edges to compute two sizes and took the mid\sphinxhyphen{}point. The resulting volume was 50.74 gallons.

\end{itemize}

\item {} 
\sphinxAtStartPar
\sphinxstylestrong{Irregular Triangular Prism}. The tank is a prism that tapers from aft to forward. We can describe this taper as a function of the distance along the fore\sphinxhyphen{}and\sphinxhyphen{}aft axis of the tank. The  resulting value was 56.9 gallons.

\end{itemize}

\sphinxAtStartPar
The simplifying assumption of congruent triangles at each end of the tank is implicit in the initial methods. This is exposed by the irregular prism computation, where we computed areas and volumes with fewer simplifying assumptions.

\sphinxAtStartPar
The use of \sphinxcode{\sphinxupquote{sympy}} makes it easy to perform the algebraic work and then confirm the formula with measured values.

\sphinxAtStartPar
The use of Jupyter Lab makes it possible to have a spreadsheet\sphinxhyphen{}like environment where we can update a measurement and see the resulting computation. We can also confirm that our computations are correct, and even adapt them as we change our assumptions.







\renewcommand{\indexname}{Index}
\printindex
\end{document}